\documentclass{beamer}
\usepackage{amsmath}
\usepackage{braket}
\usepackage{color}
\usepackage{bbold}
\usepackage{graphicx} % Required for inserting images
\usepackage{ amssymb } % for check marks
\usepackage{svg}

\usepackage{tikz}
\usetikzlibrary{quantikz2}

\usetheme{Antibes}

\title{Finding Thresholds for Steane Type Error Correction}
\subtitle{Midterm Talk}
\author{Leonard Lux}
\date{05.02.2026}

%Todo add supervisors


\begin{document}

\tikzset{ noisy/.style={fill=red!20}
}

% Title page
\begin{frame}
\titlepage
\end{frame}

% Table of Contents
\begin{frame}
\frametitle{Outline}
\tableofcontents
\end{frame}


% Motivation
\section{Motivation}
\begin{frame}{\textbf{Motivation}}
    \begin{itemize}
        \item Threshold characterizes QEC code capabilities
        \begin{itemize}
            \item depends on.... 
            \begin{itemize} 
                \item (underlying) encoding
                \item Circuit layout 
                \item Decoder
                \item Error model 
            \end{itemize}
        \end{itemize} 
        \item Steane Type Error Correction (STEC) is:
        \begin{itemize}
            \item Fault tolerant 
            \item Oneshot 
            \begin{itemize}
                \item Only one syndrome readout required 
                \item Due to redudant stabilizer readout structure
            \end{itemize}
        \end{itemize} 
        
   \end{itemize}
\end{frame}

\section{Components of the Title}
% Error Correction
\subsection{Error Correction}
\begin{frame}
    \frametitle{Basics of \textbf{Error Correction} (EC)}
    \begin{itemize}
        \item Goal: Protect quantum information against noise 
        \item My Case: Quantum memory
        \begin{itemize}
            \item Goal: Conserve arbitrary input state
            \item Under computational noise (Pauli Errors)
        \end{itemize}
    \end{itemize}
    \begin{block}{Basic Quantum Memory Circuit}
        \begin{center}
            \begin{quantikz}
                \lstick{$\ket{\Psi}_L$}&\gate[style={noisy}]{Noise}&\gate[]{Recovery}&\rstick{$\ket{\Psi }_L$}
            \end{quantikz}
        \end{center}
    \end{block}
\end{frame}

% Slide: Stabilizer Codes
% Definitions 
\subsubsection{Stabilizer Codes}
\begin{frame}{Definitions of \textbf{Stabilizer Codes}}
    \begin{itemize}
        \item Stabilizer group: 
            $ \mathcal{G} = \langle g^{(1)},...,g^{(m)}\rangle \subset \mathcal{P}_n$
        \item Codespace:  
            $\mathcal{H}_0$
        \begin{itemize}
            \item Error subspaces:
                $\mathcal{H}_s = \{ \ket{\Psi} \in \mathcal{H} \ |\ g^{(i)}\ket{\Psi} = (-1)^{s_i} \ket{\Psi} \}$ \\
            \item Syndrome:
    $s \in \{0,1 \}^m \ \text{s.t.} \ g^{(i)} \ket{\Psi}= (-1)^{s_i} \ket{\Psi} $
        \end{itemize}
        \item Centralizer:  
            $\mathcal{C}(\mathcal{G}) = \{ f\in \mathcal{P}_n | f\cdot g = g \cdot f \  \forall g \in \mathcal{G} \} $
        \item Cosets: 
            $f\mathcal{G} = \{f\cdot g|\  g\in \mathcal{G}\}$
        \begin{itemize}
            \item Cosets of Stabilizer in the Centralizer: 
            $\bar{X},\bar{Y},\bar{Z} \in \mathcal{C}(\mathcal{G}) \setminus \mathcal{G}$
            \item Cosets of Centralizer:
            $f(s)\mathcal{C}(\mathcal{G})= \mathcal{C}_{\mathbb{1}}^{s} \cup \mathcal{C}_{X}^{s} \cup\mathcal{C}_{Y}^{s} \cup\mathcal{C}_{Z}^{s} $ 
        \end{itemize}
        \item $\mathcal{P}_n$ can be written as disjoints coset of both
   \end{itemize} 
\end{frame}

% Stabilizer Coder structure
\begin{frame}{Structure of \textbf{Stabilizer Codes}}
    \begin{figure}
        \begin{center}
            \includegraphics[width=0.95\linewidth]{sketch_stabilizer_structure.pdf}
        \end{center}
    \end{figure}
\end{frame}

% Process of Error correction for surface code
% This can just be said!
% \begin{frame}{Process of Error Correction}
%     \begin{enumerate}
%         \item Measure stabilizer generators $g^{(i)}$ 
%         \item get syndrome $s$ 
%         \item decode syndrome to get correction
%         \item apply correction
%     \end{enumerate}
% \end{frame}

% ML Decoder
\subsubsection{Maximum Likelihood Decoder}
\begin{frame}{\textbf{Maximum Likelihood Decoder} (MLD)}
    \begin{itemize}
        \item Decoder: given syndrome $\rightarrow$ returns recovery operation 
        \item Maximum likelihood (ML): returns most propable recovery op. for a given syndrome 
        \begin{itemize}
            \item Defines best possible threshold
        \end{itemize}
        \item Hard task 
        \begin{itemize}
            \item Global optimization problem 
            \item Exponentially (in $n$) many possible errors 
            \item Degeneracy needs to be taken into account 
        \end{itemize}
        \item Principal (without measurement errors):  
        \textcolor{red}{Should I do a more general case}
        \begin{enumerate}
            \item given $s$ (with certainty) 
            \item determine probabilities $\pi$ of the relevant cosets of the stabilizer:
            $ \pi(\mathcal{C}) \text{ for } \mathcal{C} \in \{\mathcal{C}_{\mathbb{1}}^{s} ,\mathcal{C}_{X}^{s}, \mathcal{C}_{Y}^{s} ,\mathcal{C}_{Z}^{s} \} $ 
            \item choose recovery operation $r(s)$: 
            $r \in {C}_{ML}^s = \arg\max_\mathcal{C}(\pi(\mathcal{C})) \text{ for } \mathcal{C} \in \{\mathcal{C}_{\mathbb{1}}^{s} ,\mathcal{C}_{X}^{s}, \mathcal{C}_{Y}^{s} ,\mathcal{C}_{Z}^{s} \} $ 
        \end{enumerate}
        \item Requires noise model
    \end{itemize} 
\end{frame}

\subsubsection{Surface Code}
\begin{frame}{\textbf{Surface Code}}
    \begin{columns}
    \begin{column}{0.5\textwidth}
        \begin{itemize}
            \item Stabilizer code
            \item Transversal $\overline{\mathrm{CNOT}}$ 
            \item $[[d^2 + (d-1)^2,1,d]]$
            \begin{itemize}
                \item $m=2d\cdot(d-1)$
                \item $|s| = 2^{2d\cdot(d-1)}$
            \end{itemize}
            \item Measurement errors: \textcolor{red}{The usual way}
            \begin{itemize}
                \item $d$ rounds of syndrome measurement needed to be FT
            \end{itemize}
        \end{itemize}
    \end{column}
    \begin{column}{0.5\textwidth}  
        \begin{center}
        \includegraphics[width=0.6\textwidth]{sketch_surface_code.png}
        \end{center}
    \end{column}
    \end{columns}
\end{frame}

\subsubsection{Effective MLD}
\begin{frame}{Effective MLD}
    \textcolor{red}{Explain the proper function}
\end{frame}

\begin{frame}{Error Propagation}
    \begin{block}{$\mathrm{CNOT}$}
        \begin{center}
        \begin{quantikz}
            &\gate[1,style={noisy},]{E_X}&\ctrl{1}&& \\
            &\ghost{E_X}&\targ{}&&
        \end{quantikz}$\rightarrow$\begin{quantikz}
            &&\ctrl{1}&\gate[1,style={noisy},]{E_X}& \\
            &&\targ{}&\gate[1,style={noisy},]{E_X}&
        \end{quantikz}


        \begin{quantikz}
            &\ghost{E_Z}&\ctrl{1}&& \\
            &\gate[1,style={noisy},]{E_Z}&\targ{}&&
        \end{quantikz}$\rightarrow$\begin{quantikz}
            &&\ctrl{1}&\gate[1,style={noisy},]{E_Z}& \\
            &&\targ{}&\gate[1,style={noisy},]{E_Z}&
        \end{quantikz}

        \end{center}

    \end{block}  
    \textcolor{red}{Should I write down equations?}
    \textcolor{red}{should i focus on one type of error}
\end{frame}

\subsection{Steane Type Error Correction}
\begin{frame}{Steane Type Error Correction}
    \begin{itemize}
        \item $\neq$ Steane Code
    \end{itemize}
    \begin{block}{Steane Type Error Correction Circuits}
        \begin{center}
        % Simple Steane Type EC
        % Z-Errors
        \begin{quantikz}
            \lstick{$\ket{\Psi}_L$}&\qwbundle{n}&\targ{}&&&\gate{Z_R}&\\
            \lstick{$\ket{0}_L$}&\qwbundle{n}&\ctrl{-1}&\meter{X}&\setwiretype{c}&\gate{Decoder}\wire[u][1]{c}
        \end{quantikz}
        % X-Errors
        \begin{quantikz}
            \lstick{$\ket{\Psi}_L$}&\qwbundle{n}&\ctrl{1}&&&\gate{X_R}&\\
            \lstick{$\ket{+}_L$}&\qwbundle{n}&\targ{}&\meter{Z}&\setwiretype{c}&\gate{Decoder}\wire[u][1]{c}
        \end{quantikz}
        \end{center}
    \end{block} 
    \textcolor{red}{Show once complete circuit}
\end{frame}

\begin{frame}
    \begin{itemize}
        \item Components:
        \begin{itemize}
            \item logical ancilla qubits $\ket{0}_L,\ket{+}_L$ 
            \item logical encoding with transversal $\overline{\mathrm{CNOT}}$
            \item transversal syndrome measurement
        \end{itemize}
        \item Principal:
        \begin{itemize}
            \item Error propagation through $\mathrm{CNOT}$s 
            \begin{itemize}
                \item trivial operation on logical states
            \end{itemize}
            \item Transversal measurements $\rightarrow$ syndrome reconstruction 
       \end{itemize}
    \end{itemize}
    \textcolor{red}{Ordering makes one type of error more likely}
    \textcolor{red}{Show flow of errors through the circuit}
\end{frame}

\begin{frame}{Why is Steane Type EC interesting?}
    \begin{itemize}
    \item Advantages:
    \begin{itemize}
        \item Oneshot 
            \begin{itemize}
                \item No need for repeated syndrome measurements
            \end{itemize}
        \item Fault tolerant by design 
        \textcolor{red}{Fault tolerant is not defined yet}
        \textcolor{red}{Argue why it is fault tolerant}
    \end{itemize}
    
    \item Disadvatages:
    \begin{itemize}
        \item More (physical) qubits needed
        \item Not only next neighbour interaction 
    \end{itemize}
    \end{itemize}

    
\end{frame}


\subsubsection{Fault Tolerance}
\begin{frame}{Concept of \textbf{Fault Tolerance}}
    \begin{itemize}
        \item Motivation: Stop uncontrolled propagation of errors
        \begin{itemize}
            \item Assumption: All gates can have faults = Circuit level noise
        \end{itemize}
    \end{itemize} 
    \begin{block}{Definition Fault Tolerance}
        $x$ faults lead to an (max) weight $x$ error at the end of the codeblock
    \end{block}
    \begin{block}{Sketch: Quantum Memory Circuit}
        \begin{center}
            \begin{quantikz}
                \lstick{$\ket{\Psi}_L$}&\gate[style={noisy}]{Noise}&\gate[style={noisy}]{Noisy Recovery}&\rstick{$\ket{\Psi '}_L$}
            \end{quantikz}
        \end{center}
    \end{block}
\end{frame}

% Slide  (Error Models)

\subsubsection{Error Models}
\begin{frame}{Circuit level noise model}
    All errors of $\mathcal{O}(p)$. 
    \begin{itemize}
        \item Depolarizing single qubit noise ($Depo1$) 
        \begin{itemize}
            \item After single qubit gates 
            \item After logical encoding of qubit (Assumption!) 
        \end{itemize}
        \item Depolarizing two qubit noise ($Depo2$) 
        \begin{itemize}
            \item After $\mathrm{CNOT}$ gates
        \end{itemize}
        \item $X$-Errors ($E_X$) 
        \begin{itemize}
            \item Before $Z$-Measurements
        \end{itemize}
    \end{itemize}
\end{frame}

% Slide: Error Equations
\begin{frame}
    \begin{block}{$X$-Errors}
        \begin{equation}
            \varepsilon_{E_X}(\rho) = (1-p) \rho + p \cdot X \rho X
        \end{equation}
    \end{block}
    
    % Depo 1
    \begin{block}{Depolarizing Errors}
        \begin{equation}
            \varepsilon_{Depo1}(\rho) = (1-p) \rho + \frac{p}{3} \sum^3_{i=1} E_i \rho E_i
        \end{equation}
        \begin{equation*}
            E_i \in \{ X, Y, Z \}
        \end{equation*}

    % Depo 2
        \begin{equation}
            \varepsilon_{Depo 2}(\rho) = (1-p) \rho + \frac{p}{15} \sum_{i=1}^{15} E_i \rho E_i,
        \end{equation}
        \begin{equation*}
            E_i \in \{E_k \otimes E_l | E_k,E_l \in \{\mathbb{1},X,Y,Z\} \} \setminus \{\mathbb{1}\otimes\mathbb{1} \}
        \end{equation*}
    \end{block}
\end{frame}

% Slide: Circuit examples Errors
\begin{frame}
    \begin{block}{Example Quantum Circuits}
        \begin{quantikz}[,classical gap=0.07cm]
            \lstick{$\ket{\Psi}$}&\ctrl{1}&&&\\
            \lstick{$\ket{\Phi}$}&\targ{}&\gate[]{H}&\meter{$Z$}&\setwiretype{c}\\
        \end{quantikz}

        \begin{quantikz}[,classical gap=0.07cm]
            \lstick{$\ket{\Psi}$}&\gate[1,style={noisy},]{Dep_1}&\ctrl{1}&\gate[1,style={noisy}]{Dep_2}\wire[d][1]{q}&&&&&\\
            \lstick{$\ket{\Phi}$}&\gate[1,style={noisy},]{Dep_1}&\targ{}&\gate[1,style={noisy}]{Dep_2}&\gate[]{H}&\gate[1,style={noisy},]{Dep_1}&\gate[1,style={noisy},]{X}&\meter{$Z$}&\setwiretype{c}\\
        \end{quantikz}

        \begin{quantikz}[,classical gap=0.07cm]
            \lstick{$\ket{\Psi}$}&\ctrl{1}&&&\gate[style={noisy}]{.}&\\
            \lstick{$\ket{\Phi}$}&\targ{}&\gate[]{H}&\gate[style={noisy}]{ML-noise}&\meter{$Z$}&\setwiretype{c}\\
        \end{quantikz}

    \end{block}
\end{frame}


\subsection{Threshold}
\begin{frame}{Why is the \textbf{Threshold} interesting?}
     \begin{itemize}
        \item Def: $p_C$ physical error rate under which we can arbitrary reduce the logical error rate $p_L$
        \begin{itemize}
            \item by increasing $d$ = increasing number of qubits 
            \item above $p_C$ QEC becomes self defeating 
            \item dependent on: encoding, error model, circuit layout, decoder 
        \end{itemize}
        \item Determine:
        \begin{itemize}
            \item Using Monte Carlo simulations 
            \item Find Crossover point of different distances $d$
        \end{itemize} 
     \end{itemize}
\end{frame}

\section{Method}
\begin{frame}{Method}
    \begin{enumerate}
        \item Implement Steane Type EC with underlying surface code in stim \checkmark
        \begin{itemize}
            \item Error model: Circuit Level
            \item Assumption: FT prepared log. qubits ($\mathcal{O}(p)$)
            \item Check working using MWPM Decoder
        \end{itemize}
        \item Propagate error rate to ancillas for MLD error model \checkmark
        \item Implement ML-Decoder (not finished) 
        \begin{itemize}
            \item First version for indepenent $X-$ and $Z-$errors
            \item Determine threshold 
        \end{itemize}
        \item Upgrade to correlated $X$- and $Z-$Error ML-Decoder
    \end{enumerate}
\end{frame}

\begin{frame}{Stim implementation}
    \begin{center}
        \includegraphics[width=1\textwidth]{noisy_test.pdf}
    \end{center}  
    \textcolor{red}{Hier noch mehr relevante teile einbauen}
\end{frame}


% \begin{frame}
%     \begin{center}
%         \includegraphics[width=0.6\textwidth]{}
%     \end{center}  
% \end{frame}


% \begin{frame}
%     \begin{center}
%         \includegraphics[width=0.6\textwidth]{}
%     \end{center}  
% \end{frame}

\end{document}