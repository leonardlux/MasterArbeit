\section{Meeting}

\subsection{Papers}
\begin{itemize}
    \item QEC: An Introductionary Guide
    \item QEC for quantum memories
    \subitem up to chapter Formalism of stabilizer codes
\end{itemize}


\subsection{Topics}
\begin{itemize}
    \item Physical and logical qubits 
    \item Digitisation of quantum errors
    \item Quantum error types  
    \subitem \textcolor{red}{Is there anything special about erasure errors? }
    \subitem \textcolor{green}{Yes, there we know which qubits are affected, therefore we have more info, therfore can correct $d-1$ errors instead of only $t$ ($2$)}
    \subitem \textcolor{red}{What is special about leakage errors? Why is teleportation needed?}
    \subitem \textcolor{green}{Leakage is regarding the space, so the qubit becomes a triplet ...}
    \item correctable Errors $d=2t+1$ 
    \subitem \textcolor{red}{$t$ is the weight of the maximal recoverable error?}
    \subitem \textcolor{green}{Correct, $t$ is the maximal possible recoverable weight.} 
    \item Computational spaces/basis
    \subitem \textcolor{red}{this is just a conventional name for the basis on which we compute ?}
    \subitem \textcolor{green}{yes}
    \item  QEC code 
    \subitem \textcolor{red}{formalism: does the code include: encoder and decoder?}
    \subitem \textcolor{green}{yes and no. Yes each code has an optimized decoder, but one can use others.}
    \item Basics of code concatenation
    \item Stabilizer codes \ref{sec:basic.qc.stabilizer} 
    \subitem Stabilizer Group 
    \subitem \textcolor{red}{$\langle \rangle$, what exactly is the notation behind this, set of generators/minimal set}
    \subitem \textcolor{green}{its essentialy the same} 
    \subitem Centralizer, 
    \subitem Codespace and Error Subspace
    \subitem \textcolor{red}{Codespace is proper subset right? $\subset (\mathbb{C}^2)^{\otimes n}$}
    \subitem \textcolor{green}{Not if the codespace is trivial... so for working codes yes.}
    \subitem Recovery Operation
    \subitem \textcolor{red}{Recover by reapplying error?}
    \subitem \textcolor{green}{Yes!}
    \subitem Logical Operators, Syndrom Extraction (general), Decoder
    \subitem Maximum-Likelihood Decoding
    \subitem \textcolor{red}{How does the MLD gets to error equivalence classes from syndromes. General connection between errors (equivalence classes) and syndromes}
    \subitem \textcolor{green}{all equivalence classes have the same snydrome, also all $[E\bar{P}]$}
    \item concept of: Code thresholds
    \item concept of: fault tolerance  
    \item general concept of encoded universal computation 
    \item first introduction to surface codes (and scaling of surface codes)
    \item Just starting to be at classical Hamiltonian part of the QEC memory paper

\end{itemize}


\subsection{Topics I might need spend time on/have skipped}
\begin{itemize}
    \item General encoding circuit for stabilizer codes
    \item Error models \textcolor{green}{We assume a error model where there are uncorrelated errors occur. What is the exact name.}
    \item universal computing  \textcolor{green}{not so interisting.}
    \item Formalism of subsystem stabilizer codes
    \item much of it is in QEC memories other chapters \textcolor{green}{do the other papers first}
\end{itemize}