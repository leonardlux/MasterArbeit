\section{Basics of Quantum Error Correction (QEC)}
The goal of quantum error correction (QEC) is to preserve not a single macroscopic quantum state 
but its contained information in a macroscopic state in a subspace of an expanded Hilbert space. \cite{QECmemory}

We can destinguish between active and passive QEC (or self correction).
In the latter quantum information is encoded in physical degrees of freedom 
which are naturally protected or have litte decoherence. \cite{QECmemory}
We will focus on the active kind.

In active QEC the active gathering of error information, 
entropy is effectively removed from the computation and dumped into the ancilla degrees of freedom, 
which are supplied in known states to collect the error information. \cite{QECmemory}


\subsection{Encoding and Logical Qubit}
The specific set of instruction, 
how the information of a single physical qubit is encoded onto mulitple qubits, 
is called \textit{encoding}.
By increasing the number of qubits used to store the information (redundantly), 
we expand the Hilbert space. \cite{QECintro}

The combination of multiple qubits storing the encoded informaiton is called a \textit{logical qubit}\footnote{
    To differenciate the 'standard' qubits from the physical ones, we will call them \textit{physical qubits
    }.} 
and is noted by a subscript\footnote{
    There exists also an alternative notation, where the logical state are marked with an bar on top, 
    similar to the logical opreators $\bar{\ket{\Psi}}$. 
}
 $\ket{\Psi}_L$.


\subsection{QEC Code}
The QEC code describes how to represent the quantum information in the state space 
of many physical/elementary qubits \cite{QECmemory}.
The notational 'code' contains both an encoding, as described above, and an implied decoder.
The decoder is genereally implied to be the optimized one of each code. 
(If other options are chosen one says that they have code X with decoder Y.)


\subsection{Challenges of QEC}
\textit{Quantum} error correction codes are different to the classical error correcting codes, 
due to the properties of quantum meachanics. 
Some of the most obvious differences are...


\subsubsection{No cloning theorem}
It is \hyperref[sec:basic.math.no_cloning_theorem]{not possible} to construct a unitary operator $U_{clone}$ 
which performs the operation
\begin{equation}
    U_{clone}(\ket{\Psi} \otimes \ket{0}) \rightarrow \ket{\Psi} \otimes\ket{\Psi}.
\end{equation}
Therefore we can not just duplicate the quantum information contained in each qubit. 
This is an essential difference to classical error correction. \cite{QECintro}


\subsubsection{Both bit- and phase-flips}
QEC codes must be designed to detect both kinds ($X,Z$) of errors simultaneously! \cite{QECintro}
\textcolor{red}{
    Hier sollte ich mathematisch etwas spezifizieren warum sie simultan gemessen werden müssen 
    und welche mathematischen auswirkung es hat.}


\subsubsection{Wavefct. collpase}
We can not perform arbitrary measurments on the qubits. 
Some measurments will lead to \hyperref[sec:basic.math.wavefct_collpase]{wavefunction collapse} 
and erase encoded information. \cite{QECintro}

$\rightarrow$ stabilizer measurments



\subsection{Errors}
Before we can talk about error correction we need to understand what errors can be and how they can be treated.


\subsubsection{Digitisation of quantum errors}
\textcolor{red}{
    This does not yet explain everything I would like it to explain. 
    Why is it enough to reapply error to correct it?!} 

Due to classical bits beeing discrete there is only one type of errors (bit-flips). 
But the qubit can takes state everywhere on the Bloch sphere (continuum of states). 
Errors can therefore take also an infinte amount of forms.
For the derivation we focus on coherent (=unitary) rotations as an example for the error $E$ but it holds for arbitrary transformation
\begin{equation}
    E(\delta \theta, \delta \phi) \ket{\Psi}=
    \cos\left(\frac{\theta + \delta \theta}{2}\right) \ket{0} +
    e^{i(\phi + \delta \phi)} 
    \sin\left(\frac{\theta + \delta \theta}{2}\right) \ket{1}.
\end{equation} 

Quantum errors can be digitazied using the Pauli matricies/operator. 
To show this we express the rot. op. $U$ using the Pauli set
$\{\mathds{1},{X},{Z},{XZ}\}$
\footnote{We used that $XZ=Yc$, where $c$ is just a phase factor.}

\begin{equation}
    E(\delta \theta, \delta \phi)\ket{\Psi} = 
    e_1 \mathds{1} \ket{\Psi} +
    e_X X \ket{\Psi}+
    e_Z Z \ket{\Psi}+
    e_{XZ} XZ \ket{\Psi}
    \label{eq:error_digitization}
\end{equation}

Therefore any coherent rot.\footnote{
    This could also be generalized for arb. quantum processes
} can be decomposed into a sum from the Pauli set. 
Resulting from this QEC codes with the ability to correct errors, 
described by the $X$- and $Z$-Pauli matricies, 
will be able to correct any (coherent) error!
This is the digitasition of errors for quantum computing. \cite{QECintro}

\paragraph{For stabilizer measurments}
By applying stabilizer measurments (\textcolor{red}{
    On what do we apply the stabilizer measurments, Ancilla or Physical qubits.
    }) we force the state to collapse to a specific state. 
If we use the density matrix picture the state $E \ket{\Psi} \bra{\Psi} E^\dagger$, 
with $E$ from eq. \ref{eq:error_digitization}. 

One would normaly expect to see cross terms of the kind $X \ket{\Psi} \bra{\Psi} Z$ in the density matrix.
But if we apply the stabilizer measurments, we force the state to collapse to specific states of the form
$\sigma_i \ket{\Psi} \bra{\Psi} \sigma_i, \ \sigma_i \in \mathcal{P}_1$. 
Therefore we can just reapply the same Pauli opeartors to recover the correct state.

\href{https://quantumcomputing.stackexchange.com/questions/35211/digitization-of-errors-in-qec}{Source}
\textcolor{red}{Is this correct?(do the math!)}

\subsubsection{Quantum error types}
Base on digitasition there are two fundamental error types that can occur on each qubit, 
that need to be accounted for \cite{QECintro}
\begin{enumerate}
    \item Pauli $X$-type errors $=$ quantum bit-flips \\
    $\rightarrow$ maps: $X\ket{0} = \ket{1}; X \ket{\Psi}= \alpha X \ket{0} + \beta X \ket{1} = \alpha  \ket{1} + \beta \ket{0}   $
    \item Pauli $Z$-type errors $=$ phase-flips (no classical analogone)\\
    $\rightarrow$ maps:  $Z \ket{\Psi}= \alpha Z \ket{0} + \beta Z \ket{1} = \alpha  \ket{0} - \beta \ket{1}   $
\end{enumerate}
These errors can be corrected by reappling the same error operator.

\paragraph{Erasure Errors}
Erasure errors are errors where certain, \textit{known} qubits are lost or are known to have an error 
and are therfore considered lost. \cite{QECmemory}

\paragraph{Leakage Errors}
A leakage error occurs if a state that is not supposed to be reachable is reached. 
For example we consider a qubit system with states $\ket{0},\ket{1}$ therefore we would get a leakage error 
if the state $\ket{2}$ would be reached.
We do not take those kinds of errors into account.


\subsubsection{Code distance}\label{sec:code_distance}
The \textit{distance} $d$ is a property of this code which describes, 
the minimal number of errors that will change one 
codeword\footnote{A codeword is the set of physical qubits that describe the logical qubit} to another.
In other words, the minimal amount of errors need for an error to go undeteced. \cite{QECintro}

This definition is equivalent with the definition, 
that the distance $d$ as the minimum weight of all logical operators $\bar{P}$. 
Based on this we can define the distance as
\begin{equation}
    d = \min( |\bar{P}| )
\end{equation}
i.e. the minimum weight of any logical operator.\cite{QECmemory}


\subsubsection{Correctable Errors}
The distance is related to the total weight\footnote{
    This is identical to the number of errors if we only consider weight one errors.
    } of (random appearing) errors $t$ a code can correct \cite{QECintro}
\begin{equation}
    d = 2t+1.
\end{equation}
If erasure erros occur, we can correct up to $d-1$ erased qubits, 
due to the additional knowledge of which qubits are affected. 
We can correct the erasure error by initializing the qubits in a maximally mixed state 
and then performing the usual error correction process.\cite{QECmemory}
We can not correct leakage errors due to our assumption of qubits.

\paragraph{Derivation}
Errors with weight at most $t$, have the property that their products have weight at most $2t<d$.
Therefore the product of these operators can never be a logical operator, as those have weight $d$ or more. 
If error $E_1$ occurs and our decoder picks another error with the same syndrome $E_2$, 
we effectively get the state $E_2 E_1 \ket{\Psi}_L = E_{combined} \ket{\Psi}_L$, where both errors have weight $\leq t$.
This state has a trivial syndrome, because all parity checks/stabilizers commute with $E_{combined}$.
($E_2$ and $E_1$ have the same syndrome, therefore they either both anticommute or commute with the stabilizers.) 
Thus $E_{combined}$ having weight $2t<d$ cannot be logical operator but the parity checks only show trivial results. 
Therefore the state is still in the codespace and no logical operator could have occured.\cite{QECmemory}

In case of erasure errors we know, that at most $d-1$ errors occured.
Therefore no logical error could have occurred. 

\subsubsection{Error Model}
\textcolor{red}{Add an error model chapter to this!}


\subsection{[[n,k,d]]-Notation}\label{sssec:nkd_notation}
We characterize the QEC code by the $[[n,k,d]]$-Notation \cite{QECintro}, where 
\begin{itemize}
    \item n: total number of physical qubits per codeword/logical qubit 
    \subitem also known as block size of the code \cite{QECmemory}
    \item k: number of encoded qubits (length of the logical qubit)
    \item d: code distance
\end{itemize}


\subsection{Code Concatenation}
Code concatenation is the procedure in which we take the elementary qubits of the code words of a code $C$ 
and replace them by encoded qubits of a new code $C'$. \cite{QECmemory}
\textcolor{red}{I should add an example!}


\subsection{Code Threshold}
The \textit{threshold theorem} for stabilizer codes states,
that increasing the distance of a code results in a reduction of the logical error rate $p_L$,
if the physical error rate $p$ of individuals quibts is below a threshold $p<p_{th}$.
Therefore we have a condition for when codes can be used to supress the logical errors 
and where the process of QEC codes becomes self defeating. \cite{QECintro}

Upper bounds for the threshold probability $p_{th}$ for a given code, under a given noise model, 
can be obtained using methods from statistical meachanics.
More realistic threshold can be numerically estimated by simulating code cycles 
and decoding using efficient interference algorithms. \cite{QECintro}


\subsection{Fault tolerance}
A QEC code is said to be \textit{fault tolerant} 
if it can account for errors, of size up to the code distance, 
that occur at \textit{any location} in the circuit.
This is different from the standard assumption that the syndrom extraction and encoding operates without error
(fig. \ref{fig:basic.qc.qec_circuit}).
Converting a code to be fault tolerant may lead to addition qubits needed, 
the reptations of syndrome extraction and/or a higher code threshold. \cite{QECintro}


\subsection{Encoded computation}
A \textit{universal quantum computer} is a device that can perform any unitary operation $U$ 
that evolves a quibt register from one state to another $U\ket{\Psi} = \ket{\Psi}'$.
Any such operation $U$ can be efficetly compiled from a finite set of elementary gate, 
so called \textit{universal gate set}. \cite{QECintro}

It possible to fault tolerantly implement a subset of the gates of a universal gat set 
without having to intoduce additional qubits. 
This is achvied by a definig logical operators with a property knows as \textit{transversality}, 
which gurantees errors will not spread uncontrollably through the circuit.
A no-go theorem exists that prohibts the implementation of a full universal gate set in this way. 
Therefore alternative techniques are required, but they typically impose high cost of additional qubits. \cite{QECintro}

\textcolor{red}{What exactly is transversality?}


