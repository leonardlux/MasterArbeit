\documentclass{article}
\usepackage{graphicx} % Required for inserting images
\usepackage{braket} % imported to allow Bra-Ket formalism

\usepackage{amsmath} % added for eq and eq* (numbered and unnumbered)

\usepackage{xcolor} % added to mark questions in red


\author{leo}

\begin{document}


% Short summary and explanation of the most important technical terms
\section{Understood}

% Everything following I need to look into
\section{Needs more work!}

\subsection{Quantum Computing}
Basic concept controlling/manipultating qubits.
Quantum phenomena (such as entanglement) enable more interesting algorithms than in classical computer with classical bits.

\subsection{Qubits}
Short for 'quantum bits' and the most basic unit of a quantum computer.\footnote{Possible realizations: photons, trapped ions, superconducting circuits, spins in semiconductors etc..}

Mathematical notation using Dirac bra-ket notation.
Computational basis given by $\{\ket{0}, \ket{1}\}$ (for single qubit case).
Therefore the general qubit take the form of:
\begin{equation*}
    \ket{\Phi} = \alpha \ket{0} + \beta \ket{1}
\end{equation*}

For multi qubit systems, implicit labelling is assumed ($1,...,n$).
\begin{equation*}
    \ket{010} = \ket{0}_1 \otimes \ket{1}_2 \otimes \ket{0}_3
\end{equation*}

\subsubsection{Physical Qubits}
\subsubsection{Logical Qubits}
Logical qubits are constructed by using mulitple physical qubits in combinations with an encoding.
This is done to implement QEC. \textcolor{red}{correct?}
Logical quibits are indicated by an subscript l: $\ket{\Phi}_l$

\subsubsection{Pauli Operator Notation}


\subsection{Decoding}

\subsubsection{Maximum Likelihood Decoding}
\subsubsection{Optimal Thresshold}

\subsubsection{Efficient Optimal Decoding}

\subsection{Encoding}
Process of storing information into multiple quits (physical to logical quit transition)

\subsection{Quantum Error Correction (QEC)}
Due to external noise, errors occour in calculations. (Active) quantum error correction needed.
\subsubsection{Knill type QEC}
\subsubsection{Steune type QEC}

\subsection{Codes}
\subsubsection{Surface Codes}
\subsubsection{Stabilizer Codes}

\subsection{Gates}
\subsubsection{CNOT-Gates}
controlled-NOT gates
\subsubsection{Hadamard Gate (H)}

\subsection{Coherent Information (CI)}
of mixed state density operator, what is the equation and Physical meaning 

\subsection{Leakage}

\subsection{Qubit Loss (vs. Quantum erasure \& Quantum Error)}
\subsection{Qubit Erasure}

\subsection{Basic Math}




\end{document}