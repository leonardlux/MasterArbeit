\documentclass{beamer}
\usepackage{amsmath}
\usepackage{braket}
\usepackage{color}
\usepackage{bbold}
\usepackage{graphicx} % Required for inserting images

\usepackage{tikz}
\usetikzlibrary{quantikz2}

\usetheme{Antibes}

\title{Finding Thresholds for Steane Type Error Correction}
\subtitle{Midterm Talk}
\author{Leonard Lux}
\date{05.02.2026}

%Todo add supervisors


\begin{document}

\tikzset{ noisy/.style={fill=red!20}
}

% Title page
\begin{frame}
\titlepage
\end{frame}

% Table of Contents
\begin{frame}
\frametitle{Outline}
\tableofcontents
\end{frame}

\section{Components of the Title}

% 1. Slide (Error Correction)
\subsection{Error Correction}
\begin{frame}
    \frametitle{Error Correction (EC)}
    \begin{itemize}
        \item Goal: Protect quantum information against noise 
        \item My Case: Quantum memory
        \begin{itemize}
            \item Goal: Conserve arbitrary input state
            \item Under computational noise
        \end{itemize}
    \end{itemize}
    \begin{block}{Quantum Memory Circuit}
        \begin{center}
            \begin{quantikz}
                \lstick{$\ket{\Psi}_L$}&\gate[style={noisy}]{Noise}&\gate[]{Recovery}&\rstick{$\ket{\Psi '}_L$}
            \end{quantikz}
        \end{center}
    \end{block}
\end{frame}

% 2. Slide  (Error Models)

\subsubsection{Error Models}
\begin{frame}
    \frametitle{Error Models}    
    \begin{itemize}
        \item Pauli Errors  
        \item Circuit level noise:
        \begin{itemize}
            \item Single qubit gates \& init.: Depolarizing single qubit noise ($Depo1$) 
            \item Two qubits gates (CNOT): Depolarizing two qubit noise ($Depo2$) 
            \item Before $Z$-Measurements: $X$-Errors 

        \end{itemize}
    \end{itemize}

% % columns
%     \begin{columns}
%     \begin{column}{0.5\textwidth}

%     \end{column}
%     \begin{column}{0.5\textwidth}  %%<--- here
%         \begin{center}
%         \includegraphics[width=0.5\textwidth]{image1.jpg}
%         \end{center}
%     \end{column}
%     \end{columns}



\end{frame}

% Slide: Error Equations
\begin{frame}
    \begin{block}{Measurement Error}
        \begin{equation}
            \varepsilon_{E_X}(\rho) = (1-p) \rho + p \cdot X \rho X
        \end{equation}
    \end{block}
    
    % Depo 1
    \begin{block}{Depolarizing Errors}
        \begin{equation}
            \varepsilon_{Depo1}(\rho) = (1-p) \rho + \frac{p}{3} \sum^3_{i=1} E_i \rho E_i
        \end{equation}
        \begin{equation*}
            E_i \in \{ X, Y, Z \}
        \end{equation*}

    % Depo 2
        \begin{equation}
            \varepsilon_{Depo 2}(\rho) = (1-p) \rho + \frac{p}{15} \sum_{i=1}^{15} E_i \rho E_i,
        \end{equation}
        \begin{equation*}
            E_i \in \{E_k \otimes E_l | E_k,E_l \in \{\mathbb{1},X,Y,Z\} \} \setminus \{\mathbb{1}\otimes\mathbb{1} \}
        \end{equation*}
    \end{block}
\end{frame}

% Slide: Circuit examples Errors
\begin{frame}
    \begin{block}{Example Quantum Circuits}
        \begin{quantikz}[,classical gap=0.07cm]
            \lstick{$\ket{\Psi}$}&\ctrl{1}&&&\\
            \lstick{$\ket{\Phi}$}&\targ{}&\gate[]{H}&\meter{$Z$}&\setwiretype{c}\\
        \end{quantikz}

        \begin{quantikz}[,classical gap=0.07cm]
            \lstick{$\ket{\Psi}$}&\gate[1,style={noisy},]{Dep_1}&\ctrl{1}&\gate[1,style={noisy}]{Dep_2}\wire[d][1]{q}&&&&\\
            \lstick{$\ket{\Phi}$}&\gate[1,style={noisy},]{Dep_1}&\targ{}&\gate[1,style={noisy}]{Dep_2}&\gate[]{H}&\gate[1,style={noisy},]{Dep_1}&\gate[1,style={noisy},]{X}&\meter{$Z$}&\setwiretype{c}\\
        \end{quantikz}

    \end{block}
\end{frame}

% Slide: Stabilizer Codes
% Definitions 
\subsubsection{Stabilizer Codes}
\begin{frame}{Stabilizer Codes}
    Definitions:
    \begin{itemize}
        \item Stabilizer group: 
            $ \mathcal{G} = \langle g^{(1)},...,g^{(m)}\rangle \subset \mathcal{P}_n$
        \item Codespace:  
            $\mathcal{H}_0$
        \begin{itemize}
            \item Error subspaces:
                $\mathcal{H}_s = \{ \ket{\Psi} \in \mathcal{H} \ |\ g^{(i)}\ket{\Psi} = (-1)^{s_i} \ket{\Psi} \}$ \\
            \item Syndrome:
    $s \in \{0,1 \}^m \ \text{s.t.} \ g^{(i)} \ket{\Psi}= (-1)^{s_i} \ket{\Psi} $
        \end{itemize}
        \item Centralizer:  
            $\mathcal{C}(\mathcal{G}) = \{ f\in \mathcal{P}_n | f\cdot g = g \cdot f \  \forall g \in \mathcal{G} \} $
        \item Cosets: 
            $f\mathcal{G} = \{f\cdot g|\  g\in \mathcal{G}\}$
        \begin{itemize}
            \item Cosets of Stabilizer in the Centralizer: 
            $\bar{X},\bar{Y},\bar{Z} \in \mathcal{C}(\mathcal{G}) \setminus \mathcal{G}$
            \item Cosets of Centralizer:
            $f(s)\mathcal{C}(\mathcal{G})= \mathcal{C}_{\mathbb{1}}^{s} \cup \mathcal{C}_{X}^{s} \cup\mathcal{C}_{Y}^{s} \cup\mathcal{C}_{Z}^{s} $ 
        \end{itemize}
   \end{itemize} 
\end{frame}

% Stabilizer Coder structure
\begin{frame}{Structure of Stabilizer Codes}
    \textcolor{red}{Add image of structure here}
    \begin{figure}
        \begin{center}
            \includegraphics[width=0.8\linewidth]{sketch.jpg}
        \end{center}
    \end{figure}
\end{frame}

% Process of Error correction for surface code
% This can just be said!
% \begin{frame}{Process of Error Correction}
%     \begin{enumerate}
%         \item Measure stabilizer generators $g^{(i)}$ 
%         \item get syndrome $s$ 
%         \item decode syndrome to get correction
%         \item apply correction
%     \end{enumerate}
% \end{frame}

% ML Decoder
\subsubsection{Maximum Likelihood Decoder}
\begin{frame}{Maximum Likelihood Decoder}
    \begin{itemize}
        \item Decoder: given syndrome $\rightarrow$ returns recovery operation 
        \item Maximum likelihood (ML): returns most propable recovery 
        \item Principal:  
        \begin{enumerate}
            \item given $s \rightarrow f(s)$ 
            \item determine probabilities $\pi$ of the relevant cosets of the stabilizer:
            $ \pi(\mathcal{C}) \text{ for } \mathcal{C} \in \{\mathcal{C}_{\mathbb{1}}^{s} ,\mathcal{C}_{X}^{s}, \mathcal{C}_{Y}^{s} ,\mathcal{C}_{Z}^{s} \} $ 
            \item choose recovery operation $r$ with the highest likelihood of success: 
            $r \in {C}_{ML} = \arg\max_\mathcal{C}(\pi(\mathcal{C})) \text{ for } \mathcal{C} \in \{\mathcal{C}_{\mathbb{1}}^{s} ,\mathcal{C}_{X}^{s}, \mathcal{C}_{Y}^{s} ,\mathcal{C}_{Z}^{s} \} $ 
        \end{enumerate}
        \item Reason why it is hard? \textcolor{red}{Find some good and clean argumentation}
    \end{itemize} 
    \textcolor{red}{Add similiar picture as above to explain function}
\end{frame}

\subsubsection{Surface Code}
\begin{frame}{Surface Code}
    \begin{columns}
    \begin{column}{0.5\textwidth}
        \begin{itemize}
            \item Stabilizer code
            \item $[[d^2 + (d-1)^2,1,d]]$
            \item Transversal $\overline{\mathrm{CNOT}}$
            \item with measurement errors: $d$ rounds of syndrome measurement needed 
        \end{itemize}
    \end{column}
    \begin{column}{0.5\textwidth}  %%<--- here
        \begin{center}
        \includegraphics[width=0.5\textwidth]{sketch_surface_code.png}
        \end{center}
    \end{column}
    \end{columns}
\end{frame}

\subsubsection{Fault Tolerance}
\begin{frame}{Fault Tolerance (FT)}
    \begin{block}{Definition Fault Tolerance}
        $x$ faults lead to an (max) weight $x$ error at the end of the codeblock
    \end{block}
    \begin{itemize}
        \item Goal: Stop uncontrolled propagation of errors 
        \item Assuming: All gates can have faults = Circuit level noise
    \end{itemize} 
\end{frame}

\begin{frame}{Error Propagation Example}
    \begin{block}{$\mathrm{CNOT}$}
    \begin{columns}
    \begin{column}{0.5\textwidth}
        \begin{center}
            \begin{quantikz}
                &\gate[1,style={noisy},]{E_X}&\ctrl{1}&& \\
                &&\targ{}&&
            \end{quantikz}

            \begin{quantikz}
                &&\ctrl{1}&& \\
                &\gate[1,style={noisy},]{E_Z}&\targ{}&&
            \end{quantikz}

        \end{center}
    \end{column}
    \begin{column}{0.5\textwidth}  %%<--- here
        \begin{center}
            \begin{quantikz}
                &&\ctrl{1}&\gate[1,style={noisy},]{E_X}& \\
                &&\targ{}&\gate[1,style={noisy},]{E_X}&
            \end{quantikz}

            \begin{quantikz}
                &&\ctrl{1}&\gate[1,style={noisy},]{E_Z}& \\
                &&\targ{}&\gate[1,style={noisy},]{E_Z}&
            \end{quantikz}
        \end{center}
    \end{column}
    \end{columns}
    \end{block}  
\end{frame}

\begin{frame}{Error Propagation Example}
    \begin{block}{$\mathrm{CNOT}$}
        \begin{tikzpicture}[every node/.style={minimum height=2cm}]
        \draw[very thin, gray](0,0) grid (3,3);
            \node at () 
            \begin{quantikz}
                &\gate[1,style={noisy},]{E_X}&\ctrl{1}&& \\
                &&\targ{}&&
            \end{quantikz}

            \begin{quantikz}
                &&\ctrl{1}&& \\
                &\gate[1,style={noisy},]{E_Z}&\targ{}&&
            \end{quantikz}

            \begin{quantikz}
                &&\ctrl{1}&\gate[1,style={noisy},]{E_X}& \\
                &&\targ{}&\gate[1,style={noisy},]{E_X}&
            \end{quantikz}

            \begin{quantikz}
                &&\ctrl{1}&\gate[1,style={noisy},]{E_Z}& \\
                &&\targ{}&\gate[1,style={noisy},]{E_Z}&
            \end{quantikz}
        \end{tikzpicture}
    \end{block}  
\end{frame}

\end{document}