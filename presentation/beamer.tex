\documentclass{beamer}
\usepackage{amsmath}
\usepackage{braket}
\usepackage{color}
\usepackage{bbold}
\usepackage{graphicx} % Required for inserting images
\usepackage{ amssymb } % for check marks
\usepackage{svg}

\usepackage{tikz}
\usetikzlibrary{quantikz2}

\usetheme{Antibes}

\title{Finding Thresholds for Steane Type Error Correction}
\subtitle{Midterm Talk}
\author{Leonard Lux}
\date{05.02.2026}

%Todo add supervisors


\begin{document}

\tikzset{ noisy/.style={fill=red!20}
}

% Title page
\begin{frame}
\titlepage
\begin{center}
Supervised by:\newline 
\end{center}
\begin{center}
Luis Colmenarez, Maarten Wegewijs \& Markus Müller
\end{center}
\end{frame}

% Table of Contents
\begin{frame}
\frametitle{Outline}
\tableofcontents
\end{frame}


% Motivation
\section{Motivation}
\begin{frame}{\textbf{Motivation}}
    \begin{itemize}
        \item Threshold 
        \begin{itemize}
            \item Important charateristic of a QEC code 
            \item Is the QECC useful? 
        \end{itemize} 
        \item Steane Type Error Correction (STEC)
        \begin{itemize}
            \item Fault tolerant 
            \item One-shot/Single-shot
            \begin{itemize}
                \item Only one imperfect syndrome readout required 
                % \item Due to redudant stabilizer readout structure
            \end{itemize}
        \end{itemize} 
        
   \end{itemize}
\end{frame}

\section{Components of the Title}


% Error Correction
\subsection{Error Correction}
\begin{frame}{Basics of \textbf{Error Correction} (EC)}
    \begin{itemize}
        \item Goal: Protect quantum information against noise 
        \item This case: Quantum memory
        \begin{itemize}
            \item Goal: Conserve arbitrary input state
            \item Under computational noise (Pauli Errors)
        \end{itemize}
    \end{itemize}
    \begin{block}{Basic Quantum Memory Circuit}
        \begin{center}
            \begin{quantikz}
                \lstick{$\ket{\Psi}_L$}&\gate[style={noisy}]{Noise}&\gate[]{Recovery}&\rstick{$\ket{\Psi }_L$}
            \end{quantikz}
        \end{center}
    \end{block}
\end{frame}

% Slide: Stabilizer Codes
% Definitions 
\subsubsection{Stabilizer Codes}
\begin{frame}{Definitions of \textbf{Stabilizer Codes}}
    \begin{itemize}
        \item Stabilizer group: 
            $ \mathcal{G} = \langle g^{(1)},...,g^{(m)}\rangle \subset \mathcal{P}_n$
        \item Syndrome: $s \in \{0,1 \}^m$ 
            \begin{itemize}
                \item $\text{s.t.} \ g^{(i)} \ket{\Psi}_L= (-1)^{s_i} \ket{\Psi}_L $
            \end{itemize}
        \item Codespace:  
            $\mathcal{H}_0 \ni \ket{\Psi}_L$
        \begin{itemize}
            \item Error subspaces:
                $\mathcal{H}_s = \{ \ket{\Psi}_L \in \mathcal{H} \ |\ g^{(i)}\ket{\Psi} = (-1)^{s_i} \ket{\Psi}_L \}$ \\
        \end{itemize}
    \end{itemize} 
\end{frame}

\subsubsection{Surface Code}
\begin{frame}{\textbf{Surface Code}}
    \begin{columns}
    \begin{column}{0.5\textwidth}
        \begin{itemize}
            \item Stabilizer code
            \item Transversal $\overline{\mathrm{CNOT}}$ 
            \item $[[d^2 + (d-1)^2,1,d]]$
            \begin{itemize}
                % \item $m=2d\cdot(d-1)$
                \item  $\#s = 2^m =2^{2d\cdot(d-1)}$
            \end{itemize}
            \item Imperfect measurements:
            \begin{itemize}
                \item Standard: repeat rounds of syndrome readout 
            \end{itemize}
        \end{itemize}
    \end{column}
    \begin{column}{0.5\textwidth}  
        \begin{center}
        \includegraphics[width=0.9\textwidth]{sketch_surface_code.png}
        \end{center}
    \end{column}
    \end{columns}
\end{frame}
\begin{frame}{Structure of \textbf{Stabilizer Codes}}
    \begin{itemize}
        \item Centralizer:  
            $\mathcal{C}(\mathcal{G}) = \{ f\in \mathcal{P}_n | f\cdot g = g \cdot f \  \forall g \in \mathcal{G} \} $
        \item Cosets: 
            $f\mathcal{G} = \{f\cdot g|\  g\in \mathcal{G}\}$
        \begin{itemize}
            \item Cosets of Stabilizer in the Centralizer: 
            $\bar{X},\bar{Y},\bar{Z} \in \mathcal{C}(\mathcal{G}) \setminus \mathcal{G}$
            \item Cosets of Centralizer:
            $f(s)\mathcal{C}(\mathcal{G})= \mathcal{C}_{\mathbb{1}}^{s} \cup \mathcal{C}_{X}^{s} \cup\mathcal{C}_{Y}^{s} \cup\mathcal{C}_{Z}^{s} $ 
            \item $\mathcal{P}_n$ can be written as a disjoint union of coset (for both cases) 
        \end{itemize}
    \end{itemize}
\end{frame}

% Stabilizer Coder structure
\begin{frame}{Structure of \textbf{Stabilizer Codes}}
    \begin{figure}
        \begin{center}
            \includegraphics[width=0.9\linewidth]{sketch_stabilizer_structure.pdf}
        \end{center}
    \end{figure}
\end{frame}

% Process of Error correction for surface code
% This can just be said!
% \begin{frame}{Process of Error Correction}
%     \begin{enumerate}
%         \item Measure stabilizer generators $g^{(i)}$ 
%         \item get syndrome $s$ 
%         \item decode syndrome to get correction
%         \item apply correction
%     \end{enumerate}
% \end{frame}

% ML Decoder
\subsubsection{Maximum Likelihood Decoder}
\begin{frame}{\textbf{Maximum Likelihood Decoder} (MLD)}
    \begin{itemize}
        \item Decoder: given syndrome $\rightarrow$ returns recovery operation 
        \item Maximum likelihood (ML): returns most propable recovery op. for a given syndrome 
        \begin{itemize}
            \item Defines best possible threshold
            \item $f_R = \arg \max_{f_R} P_{success}(f_R|s)$ 
        \end{itemize}
        \item Hard task 
        \begin{itemize}
            \item Global optimization problem 
            \item Exponentially  many possible errors 
            \item Degeneracy needs to be taken into account 
        \end{itemize}
        \item Requires noise model
    \end{itemize} 
\end{frame}


\subsubsection{Specific MLD}
\begin{frame}{Specific MLD}
    \begin{center}
        \includegraphics[width=0.7\textwidth]{eff_ml_decoder_paper.png}
    \end{center}
    \begin{itemize}
        \item Designed for surface codes 
        \item Assumes:
        \begin{itemize}
            \item noiseless syndrome extraction = no measurement errors
            \item indepenent bit- \& phase-flip noise (1. version) 
        \end{itemize}
        \item Basic idea:
        \begin{itemize}
            \item Map MLD problem to matchgate quantum circuits to solve efficently ($t=\mathcal{O}(n^2)$)
        \end{itemize}
   \end{itemize}
\end{frame}

\begin{frame}{Specific MLD working principal}
    \begin{itemize}
        \item Currently studying this  
        \item Principal:
        \begin{enumerate}
            \item given $s$ (with certainty) 
            % \item determine $f(s)\in \mathcal{P}_n$ 
            \item determine probabilities $\pi$:
            $ \pi(\mathcal{C}) \text{ for } \mathcal{C} \in \{\mathcal{C}_{\mathbb{1}}^{s} ,\mathcal{C}_{X}^{s}, \mathcal{C}_{Y}^{s} ,\mathcal{C}_{Z}^{s} \} $ 
            \item choose recovery operation $f_R(s)$: 
            $f_R \in {C}_{ML}^s = \arg\max_\mathcal{C}(\pi(\mathcal{C})) \text{ for } \mathcal{C} \in \{\mathcal{C}_{\mathbb{1}}^{s} ,\mathcal{C}_{X}^{s}, \mathcal{C}_{Y}^{s} ,\mathcal{C}_{Z}^{s} \} $ 
        \end{enumerate}
        \item determines $\pi(\mathcal{C})$  efficently  using:
        \begin{itemize}
            \item MLD problem reduced to partition function of Ising-like Hamiltonian of a 2d lattice
            \item reduced to matchgate quantum circuits 
            \begin{itemize}
                \item efficently simulatable by classical computers 
            \end{itemize}
        \end{itemize}
    \end{itemize}
\end{frame}

\begin{frame}
    \textcolor{red}{Adaption to only bit or phase flip error?}
\end{frame}

\subsubsection{Fault Tolerance}
\begin{frame}{Concept of \textbf{Fault Tolerance}}
    \begin{itemize}
        \item Assumption: All gates can have faults (Circuit level noise)
        \item Motivation: Stop uncontrolled propagation of errors
        \begin{itemize}
            \item Require: Remove errors up to a correctable residual error 
            \textcolor{red}{Find a better expression?}
        \end{itemize}
    \end{itemize} 
    \begin{block}{Quantum Memory Circuit}
        \begin{center}
            \begin{quantikz}
                \lstick{$\ket{\Psi}_L$}&\gate[style={noisy}]{Noise}&\gate[style={noisy}]{Noisy-Recovery}&\rstick{$\ket{\Psi '}_L$}
            \end{quantikz}
        \end{center}
    \end{block}
\end{frame}

\begin{frame}{Error Propagation - Example}
    \begin{block}{$\mathrm{CNOT}$ and $X-Errors$}
        \begin{center}
        \begin{quantikz}
            &\gate[1,style={noisy},]{E_X}&\ctrl{1}&& \\
            &\ghost{E_X}&\targ{}&&
        \end{quantikz}$\rightarrow$\begin{quantikz}
            &&\ctrl{1}&\gate[1,style={noisy},]{E_X}& \\
            &&\targ{}&\gate[1,style={noisy},]{E_X}&
        \end{quantikz}


        \begin{quantikz}
            &\ghost{E_X}&\ctrl{1}&& \\
            &\gate[1,style={noisy},]{E_X}&\targ{}&&
        \end{quantikz}$\rightarrow$\begin{quantikz}
            &&\ctrl{1}&\ghost[1,]{E_X}& \\
            &&\targ{}&\gate[1,style={noisy},]{E_X}&
        \end{quantikz}

        \end{center}

    \end{block}  
\end{frame}


\subsection{Steane Type Error Correction}
\begin{frame}{Steane Type Error Correction - The Circuit}
    \begin{itemize}
        \item $\neq$ Steane Code
    \end{itemize}
    \begin{block}{Steane Type Error Correction Circuit}
        \begin{center}
        % Simple Steane Type EC
        \begin{quantikz}
            \lstick{$\ket{\Psi}_L$}&\qwbundle{n}&\targ{}&\ctrl{2}&&&\gate{Z_R}&\gate{X_R}&\\
            \lstick{$\ket{0}_L$}&\qwbundle{n}&\ctrl{-1}&&\meter{X}&\setwiretype{c}&\gate{Decoder}\wire[u][1]{c}\\
            \lstick{$\ket{+}_L$}&\qwbundle{n}&&\targ{}&\meter{Z}&\setwiretype{c}&&\gate{Decoder}\wire[u][2]{c}
        \end{quantikz}
        \end{center}
    \end{block} 
\end{frame}

\begin{frame}{Steane Type Error Correction - The Gadgets}
    \begin{block}{The $X$-Stabilizer Gadget}
        \begin{center}
        % Simple Steane Type EC
        % Z-Errors
        \begin{quantikz}
            \lstick{$\ket{\Psi}_L$}&\qwbundle{n}&\targ{}&&\\
            \lstick{$\ket{0}_L$}&\qwbundle{n}&\ctrl{-1}&\meter{X}&\setwiretype{c}
        \end{quantikz}
        \end{center}
    \end{block}
    \begin{block}{The $Z$-Stabilizer Gadget}
        % X-Errors
        \begin{center}
        \begin{quantikz}
            \lstick{$\ket{\Psi}_L$}&\qwbundle{n}&\ctrl{1}&&\\
            \lstick{$\ket{+}_L$}&\qwbundle{n}&\targ{}&\meter{Z}&\setwiretype{c}
        \end{quantikz}
        \end{center}
    \end{block} 
\end{frame}

\begin{frame}
    \begin{itemize}
        \item Circuit components:
        \begin{itemize}
            \item logical encoded ancilla qubits $\ket{0}_L,\ket{+}_L$ 
            \item logical encoding with transversal $\overline{\mathrm{CNOT}}$
            \item transversal syndrome measurement
        \end{itemize}
        \item Principal:
        \begin{itemize}
            \item Error propagation through $\mathrm{CNOT}$s 
            \begin{itemize}
                \item trivial operation on logical states
            \end{itemize}
            \item Transversal measurements $\rightarrow$ syndrome reconstruction 
       \end{itemize}
    \end{itemize}
\end{frame}

\begin{frame}{Why is Steane Type EC interesting?}
    \begin{itemize}
    \item Advantages:
    \begin{itemize}
        \item Oneshot 
            \begin{itemize}
                \item No need for repeated syndrome measurements
            \end{itemize}
        \item Fault tolerant by design 
    \end{itemize}
    
    \item Disadvatages:
    \begin{itemize}
        \item More (physical) qubits needed
        \item Not only next neighbour interaction 
    \end{itemize}
    \item Other points of interest:
    \begin{itemize}
        \item Influence likelihood of error types in residual error by ordering of gadgets
    \end{itemize}
    \end{itemize}
    
\end{frame}



% Slide  (Error Models)

\subsubsection{Error Models}
\begin{frame}{Circuit level noise model}
    Probability of noise is of $\mathcal{O}(p)$. 
    \begin{itemize}
        \item Depolarizing single qubit noise ($Depo1$) 
        \begin{itemize}
            \item After single qubit gates 
            \item After logical encoding of qubit (Assumption!) 
        \end{itemize}
        \item Depolarizing two qubit noise ($Depo2$) 
        \begin{itemize}
            \item After $\mathrm{CNOT}$ gates
        \end{itemize}
        \item $X$-Errors ($E_X$) 
        \begin{itemize}
            \item Before $Z$-Measurements
        \end{itemize}
    \end{itemize}
\end{frame}

% Slide: Error Equations
\begin{frame}
    \begin{block}{$X$-Errors}
        \begin{equation}
            \varepsilon_{E_X}(\rho) = (1-p) \rho + p \cdot X \rho X
        \end{equation}
    \end{block}
    
    % Depo 1
    \begin{block}{Depolarizing Errors}
        \begin{equation}
            \varepsilon_{Depo1}(\rho) = (1-p) \rho + \frac{p}{3} \sum^3_{i=1} E_i \rho E_i
        \end{equation}
        \begin{equation*}
            E_i \in \{ X, Y, Z \}
        \end{equation*}

    % Depo 2
        \begin{equation}
            \varepsilon_{Depo 2}(\rho) = (1-p) \rho + \frac{p}{15} \sum_{i=1}^{15} E_i \rho E_i,
        \end{equation}
        \begin{equation*}
            E_i \in \{E_k \otimes E_l | E_k,E_l \in \{\mathbb{1},X,Y,Z\} \} \setminus \{\mathbb{1}\otimes\mathbb{1} \}
        \end{equation*}
    \end{block}
\end{frame}

% Slide: Circuit examples Errors
\begin{frame}
    \begin{block}{Example Quantum Circuits}
        \begin{quantikz}
            \lstick{$\ket{\Psi}$}&\ctrl{1}&&&\\
            \lstick{$\ket{\Phi}$}&\targ{}&\gate[]{H}&\meter{$Z$}&\setwiretype{c}\\
        \end{quantikz}

        \begin{quantikz}
            \lstick{$\ket{\Psi}$}&\gate[1,style={noisy},]{Dep_1}&\ctrl{1}&\gate[1,style={noisy}]{Dep_2}\wire[d][1]{q}&&&&&\\
            \lstick{$\ket{\Phi}$}&\gate[1,style={noisy},]{Dep_1}&\targ{}&\gate[1,style={noisy}]{Dep_2}&\gate[]{H}&\gate[1,style={noisy},]{Dep_1}&\gate[1,style={noisy},]{X}&\meter{$Z$}&\setwiretype{c}\\
        \end{quantikz}
        \begin{center}
            Decoder error model
        \end{center}
        \begin{quantikz}
            \lstick{$\ket{\Psi}$}&\ctrl{1}&&&\gate[style={noisy}]{.}&\\
            \lstick{$\ket{\Phi}$}&\targ{}&\gate[]{H}&\gate[style={noisy}]{Pseudo-Surface-Noise}&\meter{$Z$}&\setwiretype{c}\\
        \end{quantikz}

    \end{block}
\end{frame}

\subsection{Threshold}
\begin{frame}{Why is the \textbf{Threshold} interesting?}
     \begin{itemize}
        \item The threshold $p_{th}$ is a physical error rate under which we can reduce the logical error rate $p_L$
        \begin{itemize}
            \item by increasing number of qubits $n\leftrightarrow$  increasing code distance $d$ 
            \item above $p_{th}$ QECC becomes self defeating 
            \item $p_{th}$ is dependent on: 
            \begin{itemize}
                \item logical encoding 
                \item error model
                \item circuit layout
                \item decoder
            \end{itemize}
        \end{itemize}
        \item Determine $p_{th}$:
        \begin{itemize}
            \item Monte Carlo simulations $\rightarrow p_{L}(p_{phy})$  
            \item Find intersection of $p_{L}(p_{phy})$ of different distances $d$
        \end{itemize} 
     \end{itemize}
\end{frame}
\section{Method \& some Results}
\begin{frame}{Method}
    \begin{enumerate}
        \item Implement Steane Type EC with underlying surface code in stim \checkmark
        \begin{itemize}
            \item Error model: $X-$ \& $Z-$ after initilization  \& Circuit level noise
            \item Monte Carlo simulation for different distances
            \item Assumption: FT prepared log. qubits ($\mathcal{O}(p)$)
            \item Check working using MWPM Decoder
        \end{itemize}
        \item Propagate noise model to ancillas for decoder error model \checkmark
        \item Implement decoder (not finished) 
        \begin{itemize}
            \item First version for indepenent $X-$ and $Z-$errors
            \item Determine threshold 
        \end{itemize}
        \item Upgrade to correlated $X$- and $Z-$Error decoder
    \end{enumerate}
\end{frame}

\begin{frame}{Stim implementation}
    \begin{center}
        \includegraphics[width=1\textwidth]{circuit_noisy_test.pdf}
    \end{center}  
    \textcolor{red}{Hier noch mehr relevante teile einbauen}
\end{frame}


\begin{frame}
    \begin{center}
        \includegraphics[width=1\textwidth]{circuit_noise_MWPM.pdf}
    \end{center}  
\end{frame}


% \begin{frame}
%     \begin{center}
%         \includegraphics[width=0.6\textwidth]{}
%     \end{center}  
% \end{frame}

\end{document}